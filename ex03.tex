
\addcontentsline{toc}{section}{Exercițiul 3}
\section*{3. Descrierea entitatilor, incluzand precizarea cheii primare.}

\vspace{1cm}

Pentru modelul de date referitor la lantul de mall-uri, structurile MALL, MAGAZIN, FIRMA, PRODUS, CLIENT, ANGAJAT, TRANZACTIE, RECLAMATIE, STOC,
CHIRIAS, PROMOVARE reprezinta entitati.

Toate entitatile sunt independente mai putin entitatea RECLAMATIE (care va
depinde de client si magazinul caruia i s-a atribuit aceasta reclamatie).

\subsection*{\centering ENTITATILE DE BAZA}

Acestea reprezinta substantivele preluate din descrierea bazei de date, entitatile necesare pentru a o prelucra.

\vspace{0.5cm}

\begin{enumerate}
    \item \textbf{MALL}
    - O institutie comerciala ce detine mai multe magazine. Mall-ul are scopul
    de a scoate profit din locurile comerciale inchiriate.

    \begin{itemize}
        \item CHEIE PRIMARA = \textbf{id\_mall}
        \item CHEI STRAINE 
                            \begin{itemize}
                                \item \textbf{id\_firma} care referentiaza tabela FIRMA (id\_firma)
                                \item \textbf{id\_promovare} care referentiaza tabela PROMOVARE (id\_promovare)
                            \end{itemize}
            
    \end{itemize}

    \vspace{0.5cm}
    
    \item \textbf{MAGAZIN}
    - Un spatiu comercial destinat persoanelor publice. In cadrul unui
magazin se realizeaza mai multe tranzactii de catre clienti.

    \begin{itemize}
        \item CHEIE PRIMARA = \textbf{id\_magazin}
        \item CHEI STRAINE 
                            \begin{itemize}
                                \item \textbf{id\_chirias} care referentiaza tabela CHIRIAS (id\_chirias)
                                \item \textbf{id\_mall} care referentiaza tabela MALL (id\_mall)
                            \end{itemize}
            
    \end{itemize}

    \vspace{0.5cm}

    \item \textbf{FIRMA}
    - – Reprezinta firma de securitate. Aceasta are rolul de a mentine linistea
si buna functionare a mall-ului. Acest aspect se realizeaza cu ajutorul angajatilor.

    \begin{itemize}
        \item CHEIE PRIMARA = \textbf{id\_firma}
    \end{itemize}

    \vspace{0.5cm}

    \item \textbf{PRODUS}
    - Un obiect sau serviciu vandut intr-un magazin. Acesta poate fi
achizitionat de un client in urma unei tranzactii.

    \begin{itemize}
        \item CHEIE PRIMARA = \textbf{id\_produs}
        \item CHEI STRAINE 
                            \begin{itemize}
                                \item \textbf{id\_stoc} care referentiaza tabela STOC (id\_stoc)
                                \item \textbf{id\_tranzactie} care referentiaza tabela TRANZACTIE (id\_tranzactie)
                            \end{itemize}
            
    \end{itemize}

        \vspace{0.5cm}

        \item \textbf{CLIENT}
    - Persoana care nu poate fi angajat in acelasi timp. El poate doar sa
viziteze un magazin, nu e obligat sa faca tranzactii.

    \begin{itemize}
        \item CHEIE PRIMARA = \textbf{id\_client}
    \end{itemize}

    \vspace{0.5cm}

    \item \textbf{ANGAJAT}
    - Persoana care poate fi INTERN intr-un magazin , sau PAZNIC de la
firma de securitate. Aceasta este o superentitate.

    \begin{itemize}
        \item CHEIE PRIMARA = \textbf{id\_angajat}    
    \end{itemize}

    \vspace{0.5cm}

    \begin{enumerate}
        \item \textbf{INTERN}
        - Persoana care lucreaza in interiorul unui magazin. Spre exemplu personalul de la casa de marcat sau personalul care se ocupa cu aranjatul hainelor. 

        \begin{itemize}
        \item CHEIE PRIMARA = \textbf{id\_intern}
        \item CHEI STRAINE 
                            \begin{itemize}
                                \item \textbf{id\_angajat} care referentiaza tabela ANGAJAT (id\_angajat)
                                \item \textbf{id\_magazin} care referentiaza tabela MAGAZIN (id\_magazin)
                            \end{itemize}
            
        \end{itemize}

         \vspace{0.5cm}
        
        \item \textbf{PAZNIC}
        - Persoana care lucreaza in interiorul unui mall. Acesta are scopul de a asigura linistea si siguranta clientilor. Acesta este bineinteles in stransa legatura cu firma de securitate la care lucreaza.

        \begin{itemize}
        \item CHEIE PRIMARA = \textbf{id\_paznic}
        \item CHEI STRAINE 
                            \begin{itemize}
                                \item \textbf{id\_angajat} care referentiaza tabela ANGAJAT (id\_angajat)
                                \item \textbf{id\_mall} care referentiaza tabela FIRMA (id\_firma)
                            \end{itemize}
            
        \end{itemize}
    \end{enumerate}

    \vspace{0.5cm}
    
     \item \textbf{TRANZACTIE}
    - Entitate ce contine informatii precum cine a realizat tranzactia,
metoda de plata, ce produse a cumparat, etc.

    \begin{itemize}
        \item CHEIE PRIMARA = \textbf{id\_tranzactie}
            
    \end{itemize}

    \vspace{0.5cm}

    \item \textbf{RECLAMATIE}
    - Entitate ce contine informatii precum cine a realizat reclamatia,
si motivul. Reclamatia poate fi creata doar de un client oricand acesta considera ca are un motiv pentru a face o reclamatie.

    \begin{itemize}
        \item CHEIE PRIMARA (COMPUSA) = \textbf{id\_reclamatie + id\_magazin + id\_client}
        \item CHEI STRAINE 
                            \begin{itemize}
                                \item \textbf{id\_magazin} care referentiaza tabela MAGAZIN (id\_magazin)
                                \item \textbf{id\_client} care referentiaza tabela CLIENT (id\_client)
                            \end{itemize}
            
    \end{itemize}

    \vspace{0.5cm}
    
    \item \textbf{STOC}
    - Entitate ce contine informatii precum cate produse mai sunt valabile,
numarul minim si maxim necesar de produse.

    \begin{itemize}
        \item CHEIE PRIMARA = \textbf{id\_stoc}
        \item CHEIE STRAINA
                            \begin{itemize}
                                \item \textbf{id\_magazin} care referentiaza tabela MAGAZIN (id\_magazin)
                            \end{itemize}
            
    \end{itemize}

    \vspace{0.5cm}

    \item \textbf{CHIRIAS}
    - Persoana care detine cel putin un magazin.
    
    \begin{itemize}
        \item CHEIE PRIMARA = \textbf{id\_chirias}      
    \end{itemize}

    \vspace{0.5cm}

    \item \textbf{PROMOVARE}
    - Reprezinta campania de promovare a unui mall. 

    \begin{itemize}
        \item CHEIE PRIMARA = \textbf{id\_promovare}
    \end{itemize}
    
\end{enumerate}


\subsection*{\centering ENTITATILE ASOCIATIVE}

Acestea reprezinta entitatile create in urma relatiilor de tip 3 sau de tip many-to-many


\begin{enumerate}
    \item \textbf{ACHIZITIE}
    - Aceasta entitate a fost creata in urma relatiei de tip 3 intre entitatile TRANZACTIE, MAGAZIN si CLIENT. Aceasta are scopul de a pastra informatii precum: ce contine produse contine tranzactia, cine a realizat-o si magazinul in care s-a realizat.
    
    \begin{itemize}
        \item CHEIE PRIMARA (COMPUSA) = \textbf{id\_client + id\_magazin + id\_tranzactie}      
    
    \item CHEI STRAINE
                            \begin{itemize}
                                \item \textbf{id\_client} care referentiaza tabela CLIENT (id\_client)
                                \item \textbf{id\_magazin} care referentiaza tabela MAGAZIN (id\_magazin)
                                \item \textbf{id\_tranzactie} care referentiaza tabela TRANZACTIE (id\_tranzactie)
                            \end{itemize}
    \end{itemize}
\end{enumerate}
