
\addcontentsline{toc}{section}{Exercițiul 2}
\section*{2. Prezentarea constrangerilor (restrictii, reguli) impuse asupra modelului.}
\vspace{1cm}
\begin{itemize}
    \item Nu exista entitatea lant, consideram ca este unul singur, unic.
    \item Exista mai multe mall-uri in acest lant.
    \item Toate mall-urile au cel putin un magazin
    \item O firma de securitate poate semna contractul cu mai multe mall-uri, dar un mall poate avea o singura firma de securitate.
    \item Angajatii din cadrul fiecarui magazin pot fi interni ( tin de magazin in sine) sau de la securitate.
    \item Obligatoriu fiecare magazin detine cel putin un stoc. Fiecare stoc detine produse.
    \item Pot exista mai multi client in cadrul unui magazin, dar fiecare client poate fi clientul mai multor magazine. (Relatie de tip 3)
    \item Clientii nu sunt obligati sa faca tranzactii (sa cumpere produse), si nici sa faca relcamatii.
    \item Fiecare magazin este detinut de un chirias, iar un chirias poate avea mai multe magazine.
    \item Un mall poate avea o campanie de promovare, dar nu e obligatoriu.
    \item Fiecare mall are obligatoriu o firma de securitate afiliata.
\end{itemize}
