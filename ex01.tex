
\addcontentsline{toc}{section}{Exercițiul 1}
\section*{1. Descrierea modelului real, a utilitatii acestuia si a regulilor de functionare.}

\vspace{1cm}

Modelul real va gestiona informatii legate de functionarea unui lant international de mall-uri. Baza de date are scopul de a gestiona informatiile acestor mall-uri . 

Acest lant va cupride mai multe mall-uri din tari diferite. Fiecarui mall i se va tine evidenta asupranumelui si dimensiunii acestuia (in metrii patrati). In interiorul fiecarui mall vor exista mai mute magazine. Magazinele vor emite un profit lunar (calculate in euro) si poate fi contactat prin numarul de telefon asociat. 

Pentru a asigura paza si linistea in interior,
fiecare mall va avea un contrat cu o unica firma de securitate. Firma de securitate va contine angajati responsabili cu paza. Intr-un magazin vor lucre mai multi angajati. Deci, angajatii din aceasta baza de date pot fi doar de 2 tipuri, paznici (din cadrul firmei de securitate) si interni (adica angajatii fiecarui magazin in parte). 

Fiecare magazin va avea mai multe stocuri (in baza mea de date un stoc este vazut mai mult ca o categorie de produse, entitatea produs reprezinta efectiv un singur produs, iar stocul va avea si cantitatea de produse) , care este alcatuit din mai multe produse. Produsul reprezinta bunul material pe care un client il poate procura, sau chiar un serviciu de care clientul poate beneficia. 

Clientii sunt persoane care viziteaza mall-ul pentru a face achizitii.
Acesti client sunt inregistrati in baza de date doar daca fac cel putin o achitie. Totodata, acestia pot efectua mai multe tranzactii in cadrul unui mall, in magazine diferite. Viceversa, un magazine poate vinde produse mai multor client. Orice client are optiunea de a face o reclamatie unui magazin in cazul in care ceva nu pare la locul lui. Pentru a face o relcamatie clientul are nevoie de un motiv (pe care il poate mentiona in reclamatie). 

Detinatorul unui mall se numeste chirias, si acesta poate avea mai multe
magazine inchiriate. El se ocupa cu gestionarea fiecarui magazin pe care il detine, dar si de alte aspecte financiare. 

Un mall poate avea asociata o companie de promovare, dar nu este obligatoriu. Aceasta promovare are o data cand incepe si o data cand se sfarseste. Promovarea are scopul de a face publicitate mall-ului cu telul de a atrage mai multi clienti pentru a genera mai mult profit. 
